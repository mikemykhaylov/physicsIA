\documentclass[stu, 11pt, a4paper, floatsintext]{apa7}

\usepackage[american]{babel}
\usepackage[T1]{fontenc}
\usepackage[utf8]{inputenc}

\usepackage[style=apa,backend=biber]{biblatex}
\usepackage{csquotes}
\usepackage{graphicx}
\usepackage{mathtools}
\usepackage{amssymb}
\usepackage{cleveref}

\graphicspath{{src/images/}}

\addbibresource{src/bibliography.bib}

\title{To what extent is the effectiveness of a Dyson sphere affected by the temperature of the star surrounded?}
\author{Michael V. Mykhaylov}
\authorsaffiliations{{Kolegium Europejskie}}
\course{IB: Physics SL}	
\professor{Margorzata Walczak}
\duedate{\today}

\begin{document}
	\maketitle
	Upon having started working on the laboratory investigation, I already knew it was something to do with astrophysics, as this immense field of physics has always been of interest to me. It gives me a feeling of how small in the universe we are, and how much more has humanity got to discover. Furthermore, it amazes me how, with such limited possibilities and information, scientists can work out predictions, theories and formulas that are applicable in every corner of the universe. So, when deciding the theme of the investigation, I sought something to tie together the current information with events and inventions that humanity could build in the next couple of thousand years. It was at this time when I reckoned that one of the significant constraints to the scientific progress has always been energy. From muscle power and the first fireplace to nuclear fusion and renewables, each of those steps increased the energy available to humanity to the scale unimaginable before. Currently, as we are gaining control of all of the planetary resources, we are starting to look at the sky for new sources of energy. Fortunately, humanity happens to own one — our Sun. 
	
	Nevertheless, we are currently unable to gather that energy amply. Consequently, I started researching this topic and eventually came across the concept of a Dyson sphere, a megastructure enveloping the whole star and collecting its energy. Thenceforth, I decided to do an Internal Assessment on a variation of the megastructure, the Dyson Swarm, and I calculated the effectiveness, measured by power per one solar panel satellite, of building this construction around stars of different classes and absolute magnitudes that are within reasonable distances from Earth.

	The hypothesis states that the hotter stars will emit orders of magnitude more energy. However, the type of relation between the temperature and the energy collected remains to be seen.
	\section{Background Information}
	\subsection{Selection of stars}
	For the investigation, I decided to choose six stars, one for each of the Morgan-Keenan classification system classes except for the infrequent O class. Each of them would be an average specimen of their class, with subclass numbers ranging from 4 to 5. If more than one star fits the criteria, I will choose the star with the most fitting luminosity class to its spectral class.
	\subsection{Dyson Swarm}
	A Dyson Swarm is a megastructure, consisting of a large number of satellite-like solar panels, which are orbiting a star in a dense formation and collect the star's energy \Parencite{kochai_pioneering_2020}. The advantages of this structure over the traditional Dyson Sphere are expandability and mobility. It is also relatively unobtrusive, as enough sunlight will pass to the orbiting planets. The simplest form of the Dyson Swarm is a Dyson ring, where all of the satellites share the same orbit. In this investigation, the satellites' structure would be considered a form of the surface of a spherical sector.
	\section{Investigation process}
	\subsection{Collecting Star Information}
	All of the starting data has been collected from the SIMBAD Astronomical Database \Parencite{wenger_simbad_2000}. The search was performed using the criteria queries. \Cref{fig1:query,fig2:star_data,fig3:star_measurements} demonstrate examples of star data query and the data page of a sample star. \Cref{tab1:collected_star_data} demonstrates the data collected for the stars.
	\subsection{Computing Stars Properties}
	To create the foundation of the investigation, I shall first calculate some star properties. For instance, I need the star's luminosity to calculate the power output, as well as the radius, required to calculate the orbit radius and the size of Dyson ring satellites. The input data consists of three values for each star: the distance to the star and apparent visual magnitudes of 2 different spectres of Johnson-Morgan system: Blue and Visual. These values are sufficient to calculate all the needed properties of stars.
	\subsubsection{Absolute Visual Magnitude}
	The absolute visual magnitude is the first value to be calculated. It is equal to the apparent magnitude when a stellar object is viewed from the distance of 10 parsecs, without the consideration of astronomical extinction happening due to cosmic dust and other space matter; see \cref{eq1:abs_vis_magn}.
	\begin{equation}
		\label{eq1:abs_vis_magn}
		M_v=m-5\left(\log_{10}d_{pc} - 1\right)=m-2.5log_{10}\left(\frac{d}{10}\right)^2
	\end{equation}
	\begin{equation*}
		\delta M_v=\sqrt{\frac{4.71529\delta^2d}{d^2}+\delta^2m}
	\end{equation*}
	\begin{center}
		Where $M_v$ is the absolute visual magnitude,
		
 		$m$ is the apparent visual magnitude,
 
 		$d_{pc}$ is the distance to the stellar object, $[pc]$
	\end{center}
	\subsubsection{Bolometric Correction}
	When converting from an absolute magnitude of some band to a bolometric one, one must account for the bolometric correction. It is the correction that forecasts the expected bolometric magnitude of a stellar object based on its magnitude in a specific band, usually K (visible light, for hotter stars) or V (near-infrared light, for colder stars). In the present case, the correction would be different for each star, as all of them are of different spectral classes; see \Cref{tab2:bolometric_corr}.
	\subsubsection{Absolute Bolometric Magnitude}
	The absolute bolometric magnitude is an absolute magnitude of a stellar object, which considers electromagnetic radiation on all of the spectrum, not just the one visible from the Earth; see \cref{eq2:abs_bol_magn}.
	\begin{equation}
		\label{eq2:abs_bol_magn}
		M_{bol}=M_v+BC
	\end{equation}
	\begin{equation*}
		\delta M_{bol} = \delta M_v
	\end{equation*}
	\begin{center}
		Where $M_{bol}$ is the absolute bolometric magnitude,
		
 		$BC$ is the bolometric correction
	\end{center}
	\subsubsection{Luminosity}
	It is now possible to calculate the star luminosity. It is defined to be the amount of power a star emits in a unit of time. The relationship between the luminosity and the absolute bolometric magnitude is seen in \cref{eq3:luminosity}.
	\begin{equation}
		\label{eq3:luminosity}
		M_{bol}=-2.5log_{10}\frac{L}{L_0} \therefore L=L_010^{-0.4M_{bol}}
	\end{equation}
	\begin{equation*}
		\delta L = 0.921034\sqrt{e^{-1.84207M_{bol}}L_0^2\delta^2M_{bol}}
	\end{equation*}
	\begin{center}
		Where $L$ is the luminosity, $[W]$,
		
 		$L_0$ is the bolometric correction, $3.0128 \times 10^{28}\;W$
	\end{center}
	\subsubsection{Color Index (B-V)}
	The star's colour index is required to calculate the temperature of the star. It is defined as a difference between the magnitudes of adjacent colour filters, which in our case are B and V.
	\subsubsection{Temperature}
	The effective temperature of a star is the temperature of its surface. One can obtain a good approximation of it  by considering the star a black body and using the Ballesteros' formula \Parencite{alonso_empirical_1996}; see \cref{eq4:temperature}:
	\begin{equation}
		\label{eq4:temperature}
		T=4600\left(\frac{1}{0,92(B-V)+1,7}+\frac{1}{0,92(B-V)+0,62}\right)
	\end{equation}
	\begin{equation*}
		\delta T=4600\sqrt{\left(\frac{1.08696}{(B-V+1.84783)^2} + \frac{1.08696}{(B-V+0.673913)^2}\right)^2\left(\delta^2B+\delta^2V\right)}
	\end{equation*}
	\begin{center}
		Where $T$ is the temperature, $[K]$,
		
 		$(B-V)$ is the colour index
	\end{center}
	\subsubsection{Star Radius}
	At last, all the required pieces of information are present, so the star's radius can be calculated; see \cref{eq5:star_radius}.
	\begin{equation}
		\label{eq5:star_radius}
		L=4\pi R_{st}^2\sigma T^4 \therefore R_{st}=\sqrt{\frac{L}{4\pi\sigma T^4}}
	\end{equation}
	\begin{equation*}
		\delta R_{st}=\frac{\sqrt{\frac{16L^2\delta^2T+T^2}{L\sigma T^6}}}{4\sqrt{\pi}}
	\end{equation*}
	\begin{center}
		Where $R_{st}$ is the radius of the star, $[m]$,
		
 		$\sigma$ is the Stefan-Boltzmann constant, $5,67\times10^{-8}\;Wm^{-2}K^{-4}$
	\end{center}
	\subsection{Computing Dyson Ring Properties}
	The second stage of the investigation is the computation of the properties of the Dyson ring. In this stage, the orbit radius, the total number of satellites and the surface area of the satellite will be calculated. They will be used to calculate the amount of irradiance incident on the satellites and consequently calculate the total power output of the megastructure.
	\subsubsection{Dyson Ring Orbit Radius}
	I have assumed, that if one were to build the Dyson ring in our solar system, its orbit would be in the vicinity of the orbit of Mercury. So, to obtain the radiuses of the megastructure around other stars, I wrote a proportion. In this way, the ratio between the star radius and the radius of the ring stays the same; see \cref{eq6:dyson_ring_orbit_radius}.
	\begin{equation}
		\label{eq6:dyson_ring_orbit_radius}
		\frac{R_{mc}}{R_{sun}}=\frac{R_{dr}}{R_{st}} \therefore R_{dr}=\frac{R_{st}R_{mc}}{R_{sun}}
	\end{equation}
	\begin{equation*}
		\delta R_{dr}=\delta R_{st}\frac{R_{mc}}{R_{sun}}
	\end{equation*}
	\begin{center}
		Where $R_{dr}$ is the radius of a Dyson ring, $[m]$,
		
		$R_{mc}$ is the average radius of Mercury's orbit, $5,91\times10^{10}\;m$,
		
 		$R_{sun}$ is the radius of the Sun, $6,957\times10^8\;m$
	\end{center}
	\subsubsection{Satellite Surface Area}
	In this investigation, the radius of the satellites would be considered to be 5km. The justification for this particular assumption is that it is the maximum viewing distance on Earth, should the satellites be built on an Earth-sized planet near the star. To ensure that every point of the satellite is the same distance from the star, it will also be curved, and therefore will be the same shape as the surface area of a spherical sector. \Cref{fig4:dyson_ring_satellite} demonstrates the form of the satellite.
	
	\Cref{eq7a:sector_surf_area} gives the surface area of the spherical sector (on the surface of the sphere).
	\begin{equation}
		\label{eq7a:sector_surf_area}
		\tag{7.a}
		 S=2\pi rh
	\end{equation}
	However, the cap height is currently unknown. To retrieve it, I will derive it from two formulas of the volume of the spherical sector; see \cref{eq7b:cap_height}.
	\begin{equation}
		\label{eq7b:cap_height}
		\tag{7.b}
		V=\frac{2\pi r^2h}{3}=\frac{2\pi r^3}{3}\left(1-\cos\theta\right)\;|\;:\frac{2\pi r^2}{3}
		h = r\left(1-\cos\theta\right)
	\end{equation}
	Now the only property left to obtain is the half cone angle. It is also the angle of an arc with the length of 5000m and the radius of the Dyson ring. Therefore, I decided to obtain it using the arc measure formula; see \cref{eq7c:half_cone_angle}
	\begin{equation}
		\label{eq7c:half_cone_angle}
		\tag{7.c}
		\theta = \frac{l}{r}
	\end{equation}
	Substituting \cref{eq7b:cap_height,eq7c:half_cone_angle} into \cref{eq7a:sector_surf_area}, the area of a single solar panel satellite is equal to \cref{eq7:solar_panel_area}.
	\begin{equation}
		\label{eq7:solar_panel_area}
		S_{sp}=2\pi R_{dr} \left(R_{dr}\left(1-\cos\left(\frac{R_{sp}}{R_{dr}}\right)\right)\right)=2\pi R_{dr}^2\left(1-\cos\left(\frac{R_{sp}}{R_{dr}}\right)\right)
	\end{equation}
	\begin{equation*}
		\delta S_{sp}=2\pi\sqrt{\delta^2R_{dr}\left(R_{sp}\sin\left(\frac{R_{sp}}{R_{dr}}\right)+2R_{dr}\left(\cos\left(\frac{R_{sp}}{R_{dr}}\right)-1\right)\right)^2}
	\end{equation*}
	\begin{center}
		Where $S_{sp}$ is the surface area of the solar panel, $[m^2]$
	\end{center}
	Unfortunately, as the ratio $5000/R_{dr}$ is very close to 0, the spreadsheet application was unable to perform the calculations, so an external source was used, namely Wolfram Alpha. Also, because of such small curvature of a solar panel with 5000m radius compared to billions of meters which the radius of a Dyson ring is, such complex calculations have proven themselves unnecessary, as the simple $\pi r^2$ is accurate to 12 decimal places of the output of the formula above. Nevertheless, I am glad that I derived this formula, as it can be used in further research at smaller scales.
	\subsubsection{Number of Satellites}
	The only uncalculated property of the Dyson ring is the number of the satellites surrounding the star. To calculate it, besides the diameter of one satellite, the spacing between the two of them should also be known. I consider the distance of 1000km sufficient. The reasoning is that with the said distance, in case of an outside impact, they will probably not collide. Each satellite is separated from its two neighbours by 1000km; this means the number of gaps would be the same as the number of satellites. So, the number of "satellite-gap" pairs is expressed by \cref{eq8:solar_panel_number}.
	\begin{equation}
		\label{eq8:solar_panel_number}
		N_{sp}=\left\lfloor\frac{2\pi R_{dr}}{R_{sp} + 10^6m}\right\rfloor
	\end{equation}
	\begin{equation*}
		\delta N_{sp}=\delta R_{dr}\frac{2\pi }{R_{sp} + 10^6m}
	\end{equation*}
	\begin{center}
		Where $N_{sp}$ is the number of solar panels
	\end{center}
	\subsection{Computing Energy}
	The final step of the investigation is obtaining energy measurements. In the end, I seek to retrieve the total irradiance, incident on one solar panel, which will then be used to calculate the full power input of the Dyson ring. After that, I will graph it against the temperature and discuss the type of relation found.
	\subsubsection{Satellite surface area scaled on the star}
	To determine the irradiance, one shall calculate the amount of power incident on the solar panel. However, because the star is considered a non-zero sized light source, I am unable to calculate the incident power on the satellite by utilising a simple proportion. However, if the panel is scaled from the Dyson ring radius to the star radius, the incident power would be equal to the ratio of the star surface area covered by the scaled panel times the star's luminosity. \Cref{fig5:solar_panel_scaling} demonstrates the scheme of the process. \Cref{eq9:solar_panel_area_scaled} demonstrates the formula for the scaled solar panel area
	\begin{equation}
		\label{eq9:solar_panel_area_scaled}
		\frac{S_{sp}}{S_{dr}} = \frac{S_{sps}}{S_{st}} \therefore S_{sps} = \frac{S_{sp}S_{st}}{S_{dr}} = \frac{4\pi R^2_{st}S_{sp}}{4\pi R^2_{dr}}=\frac{R^2_{st}S_{sp}}{R^2_{dr}}
	\end{equation}
	\begin{equation*}
		\delta S_{sps}=\sqrt{\frac{R_{st}^2\left(4\delta^2R_{st}S_{sp}^2R_{dr}^2+R_{st}^2\delta^2S_{sp}R_{dr}^2+4R_{st}^2S_{sp}^2\delta^2R_{dr}\right)}{R_{dr}^6}}
	\end{equation*}
	\begin{center}
		Where $S_{sps}$ is the surface area of the scaled solar panel, $[m^2]$,
		
		$S_{dr}$ is the surface area of a sphere with the radius of Dyson ring, $[m^2]$,
		
		$S_{st}$ is the surface area of the star, $[m^2]$
	\end{center}
	\subsubsection{Power Incident on the Solar Panel}
	The amount of power incident on this scaled solar panel area and the actual solar panel is the same; it is only the irradiance that will differ by the inverse square of the distance, due to the Inverse-Square law. So, as the star's luminosity is the measure of power it emits, the power emitted from this scaled area is obtained by solving a proportion; see \cref{eq10:solar_panel_input}.
	\begin{equation}
		\label{eq10:solar_panel_input}
		\frac{L}{S_{st}} = \frac{P_{sp}}{S_{sps}} \therefore P_{sp} = \frac{LS_{sps}}{S_{st}}=\frac{LS_{sps}}{4\pi R_{st}^2}
	\end{equation}
	\begin{equation*}
		\delta P_{sp}=\frac{\sqrt{\frac{S_{sps}^2R_{st}^2 + L^2\delta^2S_{sps}R_{st}^2+4L^2S_{sps}^2\delta^2R_{st}}{R_{st}^6}}}{4\pi}
	\end{equation*}
	\begin{center}
		Where $P_{sp}$ is the power incident on the solar panel, $[W]$
	\end{center}
	\subsubsection{Dyson ring total power input}
	The total power input of the Dyson ring is equal to the full irradiance times the combined surface area of all orbiting satellites. \Cref{eq11:dyson_ring_input} gives the total energy input of the megastructure.
	\begin{equation}
		\label{eq11:dyson_ring_input}
		P_i=E_eS_{sp}N_{sp}
	\end{equation}
	\begin{equation*}
		\delta P_i=\sqrt{\delta^2E_eS_{sp}^2N_{sp}^2+E_e^2\delta^2S_{sp}^2N_{sp}^2+E_e^2S_{sp}^2\delta^2N_{sp}}
	\end{equation*}
	\begin{center}
		Where $P_{i}$ is the total power input of the Dyson ring, $[W]$
	\end{center}
	\subsubsection{Dyson ring total power output}
	In the real world, no mechanism works with 100\% efficiency. Currently, the efficiency of solar panels on Earth is approximately 20\%. In space, due to the lack of atmosphere, their efficiency can reach 37,6\% \Parencite{kotamraju_modeling_2019}. \Cref{eq12:dyson_ring_output} gives the total power output of the Dyson ring.
	\begin{equation}
		\label{eq12:dyson_ring_output}
		P_o=0.376P_i
	\end{equation}
	\begin{equation*}
		\delta P_o=0.376\delta P_i
	\end{equation*}
	\begin{center}
		Where $P_{o}$ is the total power output of the Dyson ring, $[W]$
	\end{center}
	\subsubsection{Power produced by one solar panel}
	The power produced by one solar panel is the total power output divided by the number of the satellites; see \cref{eq13:solar_panel_output}.
	\begin{equation}
		\label{eq13:solar_panel_output}
		P_{spo}=\frac{P_o}{N_{sp}}
	\end{equation}
	\begin{equation*}
		\delta P_{spo}=\sqrt{\frac{\delta^2P_oN_{sp}^2+P_o^2\delta^2N_{sp}}{N_{sp}^4}}
	\end{equation*}
	\begin{center}
		Where $P_{spo}$ is the the total power output of one solar panel, $[W]$
	\end{center}
	\section{Analysing the Retrieved Data}
	\Cref{tab3:star_properties,tab4:dyson_ring_properties,tab5:energy} demonstrate the calculated values for the selected stars. \Cref{tab6:star_properties_unc,tab7:dyson_ring_properties_unc,tab8:energy_unc} demonstrate the uncertainties of calculated values for the selected stars. To find a relation between the temperature and the power incident, I have plotted the data points on the graph and drawn the best-fit lines. The best-fit line equation with  (coefficient of determination) closest to 1 and sensible data predictions would be chosen as the most accurate. \Cref{fig6:best_fit_lines} demonstrates the regressions considered, which are linear, exponential and quadratic, as, with only 6 data points, a problem of overfitting might arise with cubic and quartic formulas.
	
	Although the quadratic regression has the greatest $R^2$ of the three, it can be seen that it approaches a non-zero value with near-zero temperatures. Therefore, it cannot be considered a valid model of the solar panel power output, as no energy can be extracted from the absence of a star. As for the linear model, it has the lowest $R^2$ of the three and goes into negatives at around 4400 Kelvin, which does not make physical sense. Finally, the exponential function does get relatively close to the origin at (0, 0) and does not go into negative values. Moreover, the coefficient of determination is very close to 1. Consequently, it shall be considered the appropriate model for the relation.
	\section{Conclusion and Evaluation}
	From the relation found, a few conclusions can be drawn. Firstly, the energy collected by a single solar panel grows much quicker than the temperature, as they are related by an exponential formula. For example, when considering a pair of stars with temperatures of 4000 and 8000, respectively, the difference between the energy collected is almost eight times. Therefore, at first glance, it is much more efficient to build a Dyson Ring around a hotter, B or A-class star. However, such hot stars are scarce in the universe, and because of their temperature, their lifespan is very short, at around 40 million years. Moreover, the amount of the building material needed for constructing a structure enveloping this big of a star would be prodigious. Because of that, for a more permanent solution, a K or G class star would be a better fit. There is an abundance of them in the universe, they are much smaller, and their lifespan is immeasurably longer at around 20 to 70 billion years. They emit their energy at a slower rate, but for far longer, which makes them an excellent fit for an interstellar species, which humanity might become one day.
	
	The uncertainties of the final results are tolerable at the scales considered, with 4-18 per cent on energy collected and 0.5-2 on the star temperature. The primary source of them is the distance uncertainty, which gets bigger the further the star is from the Earth.
	
	This investigation is reliable to an extent, and its scope is mostly limited by the lack of data on other types of stellar objects. For example, I was unable to find an O, M or L class star with values and uncertainties present for their distances and magnitudes. Some other types of stars were not considered, such as T and Y class dwarves, as there is not enough information to proceed with calculations. Other inaccuracies present might be due to double-digit precision of bolometric correction and proven applicability of Ballesteros' formula only to colder stars. In the future, I would like to improve on this investigation by using new and improved data and methods.
	
	Furthermore, the investigation could have been improved by selecting more than one specimen of every star classes. For example, two more could have been chosen with 2-3 and 7-8 spectral subclasses to account for stars one the border of their classes. Expanded selection would have brought the total number of data points to 18, which would significantly raise the accuracy of the model.
	
	For further research, I created a spreadsheet document that can batch process the calculations of values and uncertainties given the star data, which allows for quick and easy investigation expansion. The graphs are updated in real-time and recreate a best-fit line with each new star added. The document gives me the ability to refine the model and make conclusions from it.
	\printbibliography
	\appendix
	\section{SIMBAD Data Retrieval}
	\begin{figure}[h!]
		\centering
		\caption{Example query for a star of B5II class}
		\includegraphics[width=1.0\textwidth]{SIMBAD_Query}
		\figurenote{This figure demonstrates an example of a query performed to gather the information about the stars studies. Distance.distance > 0 ensures we are shown only stars with known distances}
		\label{fig1:query}
	\end{figure}
	\begin{figure}[h!]
		\centering
		\caption{Example star information page}
		\includegraphics[width=1.0\textwidth]{Star_Basic_data}
		\figurenote{This figure demonstrates an example of a star information page. The information highlighted is used in the investigation. The values in square brackets are the absolute uncertainties}
		\label{fig2:star_data}
	\end{figure}
	\begin{figure}
		\centering
		\caption{Example star measurements section}
		\includegraphics[width=1.0\textwidth]{Star_Measurements}
		\figurenote{This figure demonstrates an example of a star measurements section in the information page. The information highlighted is used in the investigation. The values under err± are the absolute uncertainties}
		\label{fig3:star_measurements}
	\end{figure}
	\begin{table}
		\caption{The collected star data}
		\label{tab1:collected_star_data}
		\begin{tabular}{@{}lrrrr@{}} \toprule
			SIMBAD Name & Spectral class & Distance (pc) & \begin{tabular}[c]{@{}r@{}}Apparent visual\\ magnitude (Blue)\end{tabular} &
  \begin{tabular}{@{}r@{}}Apparent visual\\ magnitude (Visible)\end{tabular} \\ \midrule
			HD 120578    & B5II  & 969,462±37,312 & 8±0,010      & 8±0,010      \\
			HD 16769     & A5III & 131,681±1,360  & 6,059±0,014  & 5,945±0,009  \\
			HD 176095    & F5IV  & 55,135±0,091   & 6,667±0,015  & 6,207±0,010  \\
			HD 154857    & G5V   & 63,564±0,156   & 7,93±0,010   & 7,24±0,010   \\
			V* V1147 Tau & K5V   & 46,786±0,099   & 12,189±0,010 & 11±0,011     \\
			L 923-22     & M5V   & 10,507±0,009   & 13,18±0,010  & 11,676±0,010	
		\end{tabular}
	\end{table}
	\begin{table}
		\caption{The bolometric corrections for different star classes}
		\label{tab2:bolometric_corr}
		\begin{tabular}{@{}lrr@{}} \toprule
			SIMBAD Name  & Spectral class & Bolometric Correction \\ \midrule
			HD 120578    & B5II           & -1.15                 \\
			HD 16769     & A5III          & -0.14                 \\
			HD 176095    & F5IV           & -0.15                 \\
			HD 154857    & G5V            & -0.21                 \\
			V* V1147 Tau & K5V            & -0.21                 \\
			L 923-22     & M5V            & -2.73     
		\end{tabular}
		\tablenote{Table adapted from \cite{aller_landolt-bornstein_1982}}
	\end{table}
	\section{Structure schematics}
	\begin{figure}[h!]
		\centering
		\caption{The structure of the satellite making up the Dyson ring}
		\fitfigure[width=1.0\textwidth]{Spherical_Sector}
		\figurenote{This figure demonstrates the shape of the satellite, that is making up the Dyson ring. The shape of the satellite is the yellow portion of the following image. $R_{sp}$ is the radius of the solar panel $[m]$, $h$ is the height of the cap $[m]$, $\theta$ is the half-cone angle $[rad]$}
		\label{fig4:dyson_ring_satellite}
	\end{figure}
	\begin{figure}[h!]
		\centering
		\caption{The Scheme of the Solar Panel Scaling}
		\fitfigure[width=1.0\textwidth]{Solar_Panel_Scaling}
		\figurenote{This figure demonstrates the scheme of scaling down the solar panel for it to lay on the star surface. By doing so, one can calculate the power incident on the panel. Not to scale}
		\label{fig5:solar_panel_scaling}
	\end{figure}
	\section{Data Representation}
	\begin{figure}
		\centering
		\caption{Three Types of Regressions of Energy per Solar Panel versus the Star Temperature}
		\fitfigure[width=1.0\textwidth]{Best_Fit_Lines}
		\figurenote{The figure demonstrates the graphs of three best-fit lines. The dots on the graph represent the data points. First graph is a linear regression, second is the quadratic, and third is the exponential. Adapted from \cite{desmos_desmos_2020, wolfram_research_inc_wolframalpha_2020}}
		\label{fig6:best_fit_lines}
	\end{figure}
	\section{Raw values and uncertainties}
	\begin{table}[h!]
		\caption{Star Properties Calculations}
		\label{tab3:star_properties}
		\resizebox{\textwidth}{!}{
		\begin{tabular}{@{}lrrrrrr@{}} \toprule
			SIMBAD Name & \begin{tabular}{@{}r@{}}Apparent visual\\ magnitude (V)\end{tabular} & \begin{tabular}{@{}r@{}}Absolute bolo- \\ metric magnitude\end{tabular} & Luminosity, $[W]$ & Color Index & Temperature, $[K]$ & Radius of star, $[m]$ \\ \midrule
			* J Vel      & -1.933 & -3.083 & 5.153E+29 & 0.00 & 10125 & 8.295E+09 \\
			HD 16769     & 0.347  & 0.207  & 2.489E+28 & 0.11 & 8895  & 2.362E+09 \\
			HD 176095    & 2.500  & 2.350  & 3.460E+27 & 0.46 & 6576  & 1.611E+09 \\
			HD 154857    & 3.224  & 3.014  & 1.877E+27 & 0.69 & 5636  & 1.616E+09 \\
			V* V1147 Tau & 7.649  & 7.439  & 3.186E+25 & 1.19 & 4330  & 3.566E+08 \\
			L 923-22     & 11.569 & 8.839  & 8.780E+24 & 1.50 & 3787  & 2.447E+08  
		\end{tabular}
		}
	\end{table}
	\begin{table}[h!]
		\caption{Star Properties Calculations Uncertainties}
		\label{tab6:star_properties_unc}
		\resizebox{\textwidth}{!}{
		\begin{tabular}{@{}lrrrrr@{}} \toprule
			SIMBAD Name & \begin{tabular}{@{}r@{}}Apparent visual\\ magnitude (V)\end{tabular} & \begin{tabular}{@{}r@{}}Absolute bolo- \\ metric magnitude\end{tabular} & Luminosity, $[W]$  & Temperature, $[K]$ & Radius of star, $[m]$ \\ \midrule
			* J Vel      & 0.084 & 0.084 & 3.995E+28 & 176 & 2.890E+08 \\
			HD 16769     & 0.024 & 0.024 & 5.538E+26 & 156 & 8.269E+07 \\
			HD 176095    & 0.011 & 0.011 & 3.384E+25 & 87  & 4.265E+07 \\
			HD 154857    & 0.011 & 0.011 & 1.957E+25 & 49  & 2.809E+07 \\
			V* V1147 Tau & 0.012 & 0.012 & 3.497E+23 & 29  & 4.854E+06 \\
			L 923-22     & 0.010 & 0.010 & 8.229E+22 & 21  & 2.740E+06
		\end{tabular}
		}
	\end{table}
	\begin{table}[h!]
		\caption{Dyson Ring Properties Calculations}
		\label{tab4:dyson_ring_properties}
		\resizebox{\textwidth}{!}{
		\begin{tabular}{@{}lrrr@{}} \toprule
			SIMBAD Name & Dyson ring orbit radius, $[m]$ & Satellite surface area, $m^2$ & Number of satellites \\ \midrule
			* J Vel      & 7.047E+11 & 7.854E+07 & 4405430 \\
			HD 16769     & 2.007E+11 & 7.854E+07 & 1254718 \\
			HD 176095    & 1.369E+11 & 7.854E+07 & 855790  \\
			HD 154857    & 1.372E+11 & 7.854E+07 & 858068  \\
			V* V1147 Tau & 3.029E+10 & 7.854E+07 & 189372  \\
			L 923-22     & 2.079E+10 & 7.854E+07 & 129967 
		\end{tabular}
		}
	\end{table}
	\begin{table}
		\caption{Dyson Ring Properties Calculations Uncertainties}
		\label{tab7:dyson_ring_properties_unc}
		\resizebox{\textwidth}{!}{
		\begin{tabular}{@{}lrrr@{}} \toprule
			SIMBAD Name & Dyson ring orbit radius, $[m]$ & Satellite surface area, $m^2$ & Number of satellites \\ \midrule
			* J Vel      & 2.455E+10 & 5.473E+06 & 153506 \\
			HD 16769     & 7.025E+09 & 4.026E+05 & 43919  \\
			HD 176095    & 3.623E+09 & 6.134E+03 & 22652  \\
			HD 154857    & 2.386E+09 & 1.051E+04 & 14917  \\
			V* V1147 Tau & 4.124E+08 & 4.992E+03 & 2578   \\
			L 923-22     & 2.327E+08 & 3.159E+03 & 1455  
		\end{tabular}
		}
	\end{table}
	\begin{table}
		\caption{Power Properties Calculations}
		\label{tab5:energy}
		\resizebox{\textwidth}{!}{
		\begin{tabular}{@{}lrrrrr@{}} \toprule
			SIMBAD Name & \begin{tabular}{@{}r@{}}Scaled satellite\\ area, $m^2$\end{tabular} & \begin{tabular}{@{}r@{}}Incident\\ power, $W$\end{tabular} & \begin{tabular}{@{}r@{}}Total power\\ input, $W$\end{tabular} & \begin{tabular}{@{}r@{}}Total power\\ output, $W$\end{tabular} & \begin{tabular}{@{}r@{}}Power per\\ solar panel, $W$\end{tabular} \\ \midrule
			* J Vel      & 1.088E+04 & 6.486E+12 & 2.857E+19 & 1.074E+19 & 2.439E+12 \\
			HD 16769     & 1.088E+04 & 3.862E+12 & 4.846E+18 & 1.822E+18 & 1.452E+12 \\
			HD 176095    & 1.088E+04 & 1.154E+12 & 9.876E+17 & 3.713E+17 & 4.339E+11 \\
			HD 154857    & 1.088E+04 & 6.227E+11 & 5.343E+17 & 2.009E+17 & 2.341E+11 \\
			V* V1147 Tau & 1.088E+04 & 2.170E+11 & 4.109E+16 & 1.545E+16 & 8.159E+10 \\
			L 923-22     & 1.088E+04 & 1.270E+11 & 1.650E+16 & 6.205E+15 & 4.775E+10
		\end{tabular}
		}
	\end{table}
	\begin{table}
		\caption{Power Properties Calculations Uncertainties}
		\label{tab8:energy_unc}
		\resizebox{\textwidth}{!}{
		\begin{tabular}{@{}lrrrrrrr@{}} \toprule
			SIMBAD Name & \begin{tabular}{@{}r@{}}Scaled satellite\\ area, $m^2$\end{tabular} & \begin{tabular}{@{}r@{}}Incident\\ power, $W$\end{tabular} & \begin{tabular}{@{}r@{}}Total power\\ input, $W$\end{tabular} & \begin{tabular}{@{}r@{}}Total power\\ output, $W$\end{tabular} & \begin{tabular}{@{}r@{}}Power per\\ solar panel, $W$\end{tabular} & \begin{tabular}{@{}r@{}}Power per solar\\ panel, fractional, \%\end{tabular} \\ \midrule
			* J Vel      & 1.314E+03 & 9.040E+11 & 4.978E+18 & 1.872E+18 & 4.333E+11 & 17.77 \\
			HD 16769     & 1.079E+03 & 4.687E+11 & 6.131E+17 & 2.305E+17 & 1.906E+11 & 13.13 \\
			HD 176095    & 8.148E+02 & 1.058E+11 & 9.425E+16 & 3.544E+16 & 4.297E+10 & 9.90  \\
			HD 154857    & 5.351E+02 & 3.750E+10 & 3.349E+16 & 1.259E+16 & 1.523E+10 & 6.50  \\
			V* V1147 Tau & 4.191E+02 & 1.023E+10 & 2.017E+15 & 7.585E+14 & 4.156E+09 & 5.09  \\
			L 923-22     & 3.446E+02 & 4.925E+09 & 6.662E+14 & 2.505E+14 & 2.000E+09 & 4.19 
		\end{tabular}
		}
	\end{table}
\end{document}
